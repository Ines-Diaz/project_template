\section{Conclusiones}

Una de las principales dificultades ha sido encontrar el Dataset de interés para adquirir los genes asociados a la lista de proteínas del SARS-CoV2 poder realizar el correspondiente estudio de las relaciones genes-enfermedades. Esto se solucionó realizando una búsqueda exhaustiva y dando con el resultado de una lista de proteínas del SARS-CoV2 que tenían asociados ciertos genes. 

\newline

A la hora de visualizar las redes de interacciones creadas, la red creada con la librería de iGraph, es decir, la correspondiente a la relación genes-enfermades, se muestra de una forma poco visible ya que nos encontramos con que cada gen poseía única y exclusivamente una enfermedad. 

\newline

Por otro lado, de la red correspondiente a STRINGdb donde se muestra las relaciones de proteínas se puede deducir que se produce una disminución del número de asociaciones ya que se parte de interacciones entre los genes del virus, entonces al asociarse con el genoma humano, la interacción de genes-proteínas cambia porque el resto de genes van asociadas a un genoma distinto. 

\newline

Para concluir, se afirma que se aclararon un poco los resultados cuando se usó Linkcomm, ya que se puede visualizar con casi total claridad el gen que está asociado a cada enfermedad. Por otro lado, no se ha podido obtener el Link Communities y Community centrality debido a que se necesitaba una red mayor para que se consiguiera realizar la agrupación (clusters) convenientes. 

 

 
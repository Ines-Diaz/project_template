\section{Introducción}

La \textit{comorbilidad}, también conocida como \textit{morbilidad asociada}, es un término utilizado para describir dos o más trastornos o enfermedades que padece una misma persona. Pueden ocurrir simultánea o secuencialmente. La comorbilidad también implica que existe una interacción entre los dos trastornos, lo que podría empeorar la evolución de ambos.

\newline

En las poblaciones envejecidas del siglo actual, se da la situación de que muchas personas padezcan más de una enfermedad a la vez, lo que se ha convertido en uno de los principales retos a solucionar, pues, al afectar principalmente a esta población, se disminuyen las opciones de tratamiento y esperanza de vida de los pacientes. En los estudios realizados a erca de las relaciones de comorbilidad, se proporciona una visión global de la probabilidad, ya sea mayor o menor, de desarrollar enfermedades secundarias asociadas a una enfermedad previa.

\newline

De este modo, se ha podido observar que la población anciana es más susceptible a la enfermedad COVID-19 y, además, tiene una mayor probabilidad de ser ingresada en la UCI con una mayor tasa de mortalidad. 
Debido a que la COVID-19 es una enfermedad relativamente nueva, los datos de estudio disponibles son limitados. Sin embargo, se ha podido observar que las comorbilidades aumentan las posibilidades de contraer la enfermedad. En este artículo, se tratará acerca de las comorbilidades de la COVID-19.

\newline

En primer lugar, se obtuvieron las proteínas virales del SARS-CoV2 de las cuáles se recopilaron los genes de las interacciones asociadas. El objetivo principal era obtener las enfermedades vinculadas a las proteínas humanas que interaccionan con las virales, por lo que se realizaron diversas búsquedas en \textit{PheGenI} de los genes asociados para comprobar si tenían alguna enfermedad asociada o no. Todos los genes estaban relacionados con diferentes procesos en el organismo humano, pero no fueron tantos los vinculados a enfermedades.

\newline

A lo largo de este artículo, se desarrollarán los procedimientos tomados para determinar las comorbilidades de la COVID-19, es decir, para concretar cuáles son las enfermedades que provocan en las personas que las padecen más susceptibilidad a contraer la infección por SARS-CoV2.
